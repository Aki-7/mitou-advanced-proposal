%!TEX encoding = UTF-8
\documentclass[uplatex]{jsarticle}
\usepackage[dvipdfmx]{graphicx}
\usepackage{hyperref}
\usepackage{url}
\usepackage{titlesec}
\usepackage[top=2cm, bottom=2cm, left=1.5cm, right=1.5cm]{geometry}

\begin{document}

\begin{flushright}
  【様式4】
\end{flushright}
テーマ名: ``ZIGEN: 3D Windowing System'' \\
申請者名: 木内 陽大、江口 大志


\section{プロジェクトの背景,目的,目標}

% 提案の背景、目的、目標を、開発と事業化(または社会課題の解決に寄与するような社会実装)の
% 両面について、その分野の専門家でない人にもわかるように丁寧に記述してください。
% また、競合する事業等が存在する場合には、その概要及び競合との差別化要因になるものについて、
% 記載してください。

% 提案の背景(なぜ)、目的(なにを)、目標(どこまで)を、
% 開発と事業化(または社会課題の解決に寄与するような社会実装)の両面で記載
% 競合する事業等が存在する場合には、その概要及び競合との差別化要因

\section{開発に関する未踏性の主張,期待される効果}

% どのような斬新で独創的なアイディアに基づいて研究開発を進めていく技術なのか、保有す
% る技術シーズの優位性を説明してください。また、研究開発した技術結果を利用するとどのよ
% うな効果が期待できるのか説明してください。

\section{具体的な進め方}

% ・計画の緻密さを確認するため、以下の項目を記述してください。
% - 現状のプロトタイプ(あれば)
% - 事業期間中の開発内容
% - 開発体制:目標を達成できる体制になっているか。チームの場合はコミットされた役割
% 分担等も記載してください。他の未踏事業に応募しているメンバーがいる場合、その
% メンバーが抜けた場合の体制についても記載してください。
% - 開発者スキル:プログラミング等開発を行うために必要なスキルをもっているか。また、
% IPA 未踏 IT 人材発掘・育成事業の修了生である場合や、スーパークリエータの認定を受けてい
% る場合はその旨を記述してください。
% - 開発線表 (スケジュール)
% - 克服すべき課題とその解決策

\section{事業化・社会実装の新規性・優位性・想定するターゲットと規模}

% ・本提案で開発する製品・サービスの新規性・優位性等、対象とする市場、社会課題またはユー
% ザについて記述してください。また、ニーズの大きさ(市場規模やユーザ規模)についても具
% 体的に記述してください。競合サービスおよび類似プロジェクト等がある場合は、本提案との
% 比較(相違点、優位点等)を記述してください。

\section{事業化・社会実装の具体的な進め方}

% ・事業化・社会実装の進め方について、以下の項目を記述してください。
% - 本提案で開発する製品・サービスのビジネスモデルまたは社会実装のサービスモデル
% - 事業期間中の事業化・社会実装に向けての作業内容(事業化体制の構築、マーケティン
% グ等)
% - 作業体制:目標を達成できる体制になっているか。チームの場合はコミットされた役割
% 分担等も記載してください。他の未踏事業に応募しているメンバーがいる場合、その
% メンバーが抜けた場合の体制についても記載してください。
% - 作業線表(スケジュール)
% - 克服すべき課題とその解決策

\section{事業期間終了後の事業化・社会実装に関する計画}

% ・本提案で開発した製品・サービスによる、事業期間終了後の人員計画、経費計画、売上計画
% (または社会実装での実用化計画)などを記述してください。

\end{document}
