\section{事業化・社会実装の新規性・優位性・想定するターゲットと規模}

% ・本提案で開発する製品・サービスの新規性・優位性等、対象とする市場、社会課題またはユー
% ザについて記述してください。また、ニーズの大きさ(市場規模やユーザ規模)についても具
% 体的に記述してください。競合サービスおよび類似プロジェクト等がある場合は、本提案との
% 比較(相違点、優位点等)を記述してください。

ZIGENの社会実装における新規性は,\ref{section:dev-effect}章で述べたコンセプトの
新規性であり,優位性は未踏IT人材発掘・育成事業においてクオリティの高いプロトタイプが
すでに完成している点である.

対象とするユーザは,本プロジェクトの目標の1つである,XR Windowing Systemの概念を社会に
インストールするという意味では,既存の2D Windowing SystemのユーザつまりデスクトップPCの
ユーザと同じで最大40億人以上であり,今回提案しているXR Windowing Systemが
十数年後の世界で誰もが持っているARグラスやMR作業環境において基礎的な技術となっていることを
目指している.

直近のユーザを考えると,Linux Desktopユーザは世界に4500万人ほどであると推定されており,
またIDCの2022年3月のレポート(図\ref{fig:idc})から,2021年はAR/VRヘッドセットが
1000万台以上出荷され,2022年度は1500万台を超えると予想されており,この共通集合が
ターゲットとなる.少ないように思われるが,本プロジェクトが第一にターゲットとしたい
XR世界を作り上げているコミュニティではHMDを持っていることは当然のことながら,
Linuxをすでに保持していたり,デュアルブートなどで立ち上げることはそこまでコストが高くない.
2021年にSteamVRがmacOSをサポート対象から外したが,Linuxについては
SteamVRもOpenXRなどもサポート・開発を続けており,XRにおいて特定の企業に依存しない
Linuxの重要性は大きいと言える.

また,少し先の話になるが,本プロジェクトの成果物が十分にLinuxでの作業空間として有効性が
社会に認められた場合,マルチディスプレイ環境を求めるユーザにとって,
複数枚のディスプレイを購入し,物理的に部屋に配置するコストや実現性と,HMDだけを購入し,
XR空間で自由に何枚でもディスプレイを配置するコストを考えた場合,金銭的にも
部屋や机などの物理的な制約的にも,XR Windowing Systemを用いたほうが優位に働きうるため,
4500万人のLinuxユーザの大部分がターゲットになる.(例えば3万円のディスプレイ5枚より,
Valve Indexの方が安価であり.Oculus Quest2は4万円しない.)

類似プロジェクトとの差分は主にそのコンセプト・目的の違いであり,その詳細は
\ref{section:project-uniqueness}を参照されたい.
