\section{事業化・社会実装の新規性・優位性・想定するターゲットと規模}
\label{section:biz-effect}

% ・本提案で開発する製品・サービスの新規性・優位性等、対象とする市場、社会課題またはユー
% ザについて記述してください。また、ニーズの大きさ(市場規模やユーザ規模)についても具
% 体的に記述してください。競合サービスおよび類似プロジェクト等がある場合は、本提案との
% 比較(相違点、優位点等)を記述してください。

ZIGEN Windowing Systemの新規性・優位性は, 第\ref{section:project}章で述べた通り, VR空間で複数の既存の2Dアプリケーションに加えて、複数の3Dアプリケーションを表示できて、適切に入力を割り振り、アプリケーション間でデータを共有できる点にある.
ZIGEN Windowing System上では, 1つの空間内で、個々の機能を個人レベルから企業の開発者が開発することができ, 個人にとって最適な作業空間を使える.

このZIGEN Windowing Systemを社会に提供するにあたり、大きく3つのソフトウェアの提供のシナリオが考えられる.
\begin{enumerate}[label=(\alph*)]
  \item \label{biz-effect:ws} ZIGEN Windowing Systemのみの提供
  \item \label{biz-effect:os} ZIGEN Windowing Systemを内包したOS(Linux distro)の提供
  \item \label{biz-effect:pc} 上記OSをプリインストールしたホストPCやHMDの提供
\end{enumerate}

Linux環境のホストPCを持つユーザーには\ref{biz-effect:ws}のWindowing Systemのレイヤのみを提供し、Linux環境以外のホストPCを持つユーザーには\ref{biz-effect:os}のWindowing Systemに加えてLinux Kernelやファイルシステムなどを含むOSとして提供、ハードウェアごと導入したいユーザーには\ref{biz-effect:pc}のOSをプリインストールしたホストPCやHMDなどのハードウェアを提供する.

\ref{biz-effect:os}のOSとしての提供や\ref{biz-effect:pc}のハードウェアとしての提供は, ユーザーがLinux環境を用意する必要がない反面、OSのインストールやハードウェアの購入などセットアップの時間的・金銭的コストがかかってしまう.
したがって, 未踏アドバンスト期間では\ref{biz-effect:ws}のWindowing Systemのみを先に提供できるようにしてユーザー体験の検証を行い、\ref{biz-effect:os}や\ref{biz-effect:pc}は未踏アドバンスト期間終了後に推し進めていく.

\ref{biz-effect:ws}のユーザとしてLinux Desktopの全世界のユーザー数はおよそ4500万人で、この内ZIGENが対応しているVRデバイスを所持しているユーザーが最初の対象となる.
% TODO: 必要: 出典
特にVR環境でデスクトップの作業が行いたい人(既存のサービスとしてVirtual Desktopやbigscreenを使っているユーザー)や, VR環境内でVRのアプリケーションを開発したいユーザーなどに興味を持ってもらえる可能性があると考えている.
またLinux Desktopをまだ持っておらずWindows OSなどの環境を利用しているユーザーでもdual bootなどを通じてZIGEN Windowing Systemを利用可能にできるため, Windows OSのユーザーに対してもセットアップの手順を提示する.

未踏アドバンスト期間でLinux Desktopを使えるユーザーでのユーザーテストやユーザーインタビューを終え, 開発コミュニティもある程度広げた後に、上記で示した\ref{biz-effect:os}のOSとしての提供や\ref{biz-effect:pc}のハードウェアと共に提供する形を検証する.
これらのターゲットとなるユーザーは膨大で, 
% Windows OSの市場全体・VRのハードウェアの市場全体・etc...

本提案のシステムはネットワーク効果が働くプロジェクトである.
具体的には, ユーザー数が増えれば増えるほど, ZIGENのアプリケーションの開発者が増え、より高品質なアプリケーションを使えるようになる.
それらはユーザーを増やす一助となる.
こうした好循環のサイクルのきっかけを
