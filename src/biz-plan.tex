\section{事業化・社会実装の具体的な進め方}

% ・事業化・社会実装の進め方について、以下の項目を記述してください。
% - 本提案で開発する製品・サービスのビジネスモデルまたは社会実装のサービスモデル
% - 事業期間中の事業化・社会実装に向けての作業内容(事業化体制の構築、マーケティン
% グ等)
% - 作業体制:目標を達成できる体制になっているか。チームの場合はコミットされた役割
% 分担等も記載してください。他の未踏事業に応募しているメンバーがいる場合、その
% メンバーが抜けた場合の体制についても記載してください。
% - 作業線表(スケジュール)
% - 克服すべき課題とその解決策
