\section{事業化・社会実装の具体的な進め方}

% ・事業化・社会実装の進め方について、以下の項目を記述してください。
% - 本提案で開発する製品・サービスのビジネスモデルまたは社会実装のサービスモデル
% - 事業期間中の事業化・社会実装に向けての作業内容(事業化体制の構築、マーケティン
% グ等)
% - 作業体制:目標を達成できる体制になっているか。チームの場合はコミットされた役割
% 分担等も記載してください。他の未踏事業に応募しているメンバーがいる場合、その
% メンバーが抜けた場合の体制についても記載してください。
% - 作業線表(スケジュール)
% - 克服すべき課題とその解決策

第\ref{section:biz-effect}章で述べたとおり, 製品の提供の形が複数考えられる.
したがってそれらの

未踏アドバンスト期間中は, Linux Desktop環境が使えるユーザーをターゲットとし, Windowing Systemのみを提供する形に絞る.
これらのユーザーに認知され、実際に試してもらうために、未踏アドバンスト期間中に以下のことを行う.

\subsection{世界への周知}
後述の、リリース時に実際に使ってもらったり、ユーザーテストやユーザーインタビューに参加してもらったり、ユーザと開発者からなるエコシステムを活性化したりする上で最も重要なことの1つが世界にZIGENを認知してもらうことであると考えている.
そのため開発と並行して, 以下のような場でアウトプットを行っていく(メディアなどは取り上げてもらうために担当者に連絡する).
\begin{itemize}
  \item メディア: Mogura VR, Hacker News, ITmedia, GIGAZINE, @ITなど
  \item カンファレンス: XRコンソーシアム, XR Kaigi, 技育祭(発表済み), DIGITAL CONTENT EXPO, VIVECONなど
  \item 国内/国外学会: SIGGRAPH(投稿済み, レビュー待ち), CHI, WISSなど
  \item インフルエンサへのアプローチ: YouTuber, VTuberなど
  \item スタートアップ関係: PR Times, IVS LAUNCHPADなど
  \item その他: Twitter, Discord, メールなど
\end{itemize}

これら周知のための活動は事業期間中絶えず行っていき、エコシステムを大きくしていきたい.
ZIGENは日本のみでなく世界へも周知したいと考えているため, 国内外問わずアウトプット活動をすると共に, ZIGENのサイトやセットアップのためのガイド, 開発用ドキュメントなどは全て英語のものも用意する予定である.

これらの広報活動のうち効果的なものを見極めるため, 各期間ごとの
\begin{itemize}
  \item Windowing Systemのインストール数
  \item ホームページ・ドキュメントのインプレッション数
  \item GitHubのスター数/PR数
  \item Twitterのフォロワー数
  \item Discordの参加者数
\end{itemize}
などZIGENの世間への認知度やエコシステムの規模を把握するための指標を監視して広報活動や未踏アドバンスト期間の成果の測定に役立てる.

\subsection{ユーザーインタビュー/ユーザーテスト}

\subsection{リリース}
現在できている
