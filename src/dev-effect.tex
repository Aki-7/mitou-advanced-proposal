\section{開発に関する未踏性の主張,期待される効果}

本プロジェクトの中核の技術はWindowing SystemをXR用に再定義し,プロトコルの設計から行い,
ディスプレイサーバを実装していることである.\ref{section:forrest}で述べた論文では,
観測する限り唯一3D Windowing Systemの概念が提唱されているが,Windowing Systemとして
必要な多くの機能(アプリケーションのリサイズや移動など)が未定義であり,ユーザが使える状態まで
実装されてはいない.

\ref{section:current-status}で述べた,既存の2Dアプリケーションと3Dアプリケーション
との間のドラッグ \& ドロップや,既存の2DアプリケーションとHMDとの完全なフレーム同期などを
実現するには,2DのWindowing Systemの中身を理解し,ZIGENで定義するプロトコルを
2DのWindowing Systemとうまく接続できるように設計する必要がある.
これには Windowing Systemに対する理解を深め,かつOpenGLを用いたレイヤの低い3DCGと
融合する必要があり,この部分に関して自分たち以上に知見を持っており,
取り組んでいる人はいないと考える.

この技術により期待される効果は単にユーザからみて,好きなアプリケーションを
選んで使えるというだけにとどまらない.
現状のような1つのアプリケーションが全ての機能を実装する仕組みでは,ニーズの割合が小さい
例えば医療CTスキャンの結果を表示するようなアプリケーションが,ブラウザなどのような
他の作業空間としての機能と同時に開発されるハードルはかなり高くなってしまう.
ZIGENのマルチアプリケーション・マルチタスクの環境が提供されると,特定の機能だけ実装すれば,
他のアプリケーションと同時に使えるため,個人レベルの開発者であっても誰でもよりオープンに
XR環境を改善できるようになる.
また,現状では作業空間を提供するアプリケーションは,多機能であることが大きな市場価値の
ひとつとなっているが,ZIGENの環境では機能ごとに市場競争がうまれ,XR環境の市場の多様性が
向上すると共に,アプリケーションの品質が市場価値となり,XR空間がより洗練されたものに
なってゆくだろう.

% どのような斬新で独創的なアイディアに基づいて研究開発を進めていく技術なのか、保有す
% る技術シーズの優位性を説明してください。また、研究開発した技術結果を利用するとどのよ
% うな効果が期待できるのか説明してください。
