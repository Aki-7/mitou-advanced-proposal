\section{事業期間終了後の事業化・社会実装に関する計画}

事業期間終了後もOSSとして開発を続けていくつもりである.
ただし来年度からすぐにビジネスとして収益化できる可能性は現時点では低いため,
助成やFoundationへの申請によってOSSにフルタイムでコミットできるようになるのが
直近の目標となる.その後さらなるサービスの展開のために,会社を立てて人員を増やし,
先に述べたような方法で収益化を考えるか,VRヘッドセットも開発してるSonyの研究所である
Sony CSL\footnote{Sony CSL:https://www.sonycsl.co.jp/}
のような場所で研究開発として実用度合いを高めていくような進路も考えている.

% ・本提案で開発した製品・サービスによる、事業期間終了後の人員計画、経費計画、売上計画
% (または社会実装での実用化計画)などを記述してください。
