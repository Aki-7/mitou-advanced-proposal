\subsection{現状の開発成果}
\label{section:current-status}

XRでのマルチアプリケーション・マルチタスクの作業環境を実現するためには,
複数のアプリケーションの描画情報を1つの空間に合成して表示したり,
入力デバイスからの入力を適切なアプリケーションに割り振ったりする,
2D デスクトップ環境でのWindowing Systemの機能が必要となる.
そこで本プロジェクトでは未踏IT育成人材・発掘事業の支援を受けてXR向けのWindowing System,
"ZIGEN"を開発してきた.

ZIGENではWindowing SystemのライブラリであるWaylandの上で,XR用にディスプレイサーバと
アプリケーション間のプロトコルを定義し,参照実装としてのディスプレイサーバ, "ZEN" を実装した.
また既存の2Dアプリケーション(Google Chromeなど)を表示するためのアプリケーションである
"zmonitors"やその他サンプルのアプリケーションを実装した.
これによって以下のことができるようになっている.

\begin{enumerate}
  \item 空間の一部を占めるような複数のアプリケーションをそれぞれ起動し,
        同時に表示する(図\ref{fig:multi-app}).\\
        それぞれのアプリケーションはターミナルなどからプロセスとして立ち上げ,
        % textlint-disable
        OpenGLに似たAPIのZIGENプロトコルに従って頂点バッファ・頂点配列・テクスチャ・
        シェーダ(GLSL)・その他オプションをディスプレイサーバに伝えることで描画を行う.
        % textlint-enable
        ディスプレイサーバ側では受け取った描画情報を各アプリケーションどうしの
        前後関係などが正しく考慮された形で合成し,1つの空間に描画している.
        各アプリケーションはOpenGLで直接描画する場合とほとんど同じインターフェースで
        描画できるため,描画の自由度は大きく損なわない.また他のアプリケーションのことを
        知らずに描画が行え,開発元の違うアプリケーションどうしでも自然に重ね合わされる.
  \item Rayとキーボードを用いて適切に入力をアプリケーションに伝えられる
        (図\ref{fig:ray-input}). \\
        ZIGENではRay(半直線)とキーボードを用いてアプリケーションを操作する.
        Rayは2Dデスクトップのポインタに相当し,アプリケーションへのクリックやスクロール,
        ドラッグといった操作を与える.さらに,Rayによってアプリケーションへ
        フォーカスでき,キーボードイベントはフォーカスしたアプリケーションへのみ渡される.
  \item 既存の2Dアプリケーションを利用できる(図\ref{fig:2d-apps}).
        Google Chromeなどの既存の2Dアプリケーションが修正なしでそのまま動作し,
        Rayを用いて操作できる(Rayと2Dウィンドウとの交点がカーソルとして機能する.).
        この機能を提供するために作成した"zmonitors"というソフトウェアは画面共有のような
        仕組みを用いるのではなく,図\ref{fig:zmonitors}のように2Dのディスプレイサーバとして
        Google Chromeなどと2D Windowing System(Wayland)のプロトコルでやりとりを
        行いつつ,かつ3DアプリケーションとしてZIGENディスプレイサーバとやりとりを行い,
        3D空間上に2Dアプリケーションを表示している.画面共有やVNCなどではなく,
        ディスプレイサーバから実装することによって,画面更新をより効率的に行えたり,
        ヘッドマウンテッドディスプレイと2Dアプリケーションが完全にフレーム同期を行えたりする
        利点がある.これを実現するためにZIGENのプロトコルは既存の
        2D Windowing Systemと変換可能に設計されている.
  \item 2Dと3Dの垣根を越えたドラッグ \& ドロップ(図\ref{fig:dnd}). \\
        ZIGENでは既存の2Dアプリケーションと3Dアプリケーションとの間でドラッグ \& ドロップが
        可能である.このようなことが可能であるのもZIGENが2DのWindowing Systemと
        変換可能に設計され,zmonitorsが2Dディスプレイサーバから実装している利点であり,
        このような2Dと3D アプリケーション間のドラッグ \& ドロップを実装したのは
        調べる限り世界で初である.
\end{enumerate}

開発成果物のクオリティなどに関しては是非"http://..."にアップロードしたデモ動画をご覧
いただきたい. % TODO: Upload デモ


\begin{figure}[htbp]
  \begin{minipage}[t]{0.50\linewidth}
    \centering
    \includegraphics[keepaspectratio, width=\linewidth]{fig/multi-app.png}
    \caption{
      複数3Dアプリケーションの表示.左は既存の2Dアプリケーション(Google Chrome)
      中心はサンプルで作成した天体を表示・編集するアプリケーション,
      右は3Dファイルを表示するアプリケーション.また背景の空も1つのアプリケーションであり,
      ユーザが任意に変更可能である.
    }
    \label{fig:multi-app}
  \end{minipage}
  \begin{minipage}[t]{0.50\linewidth}
    \centering
    \includegraphics[keepaspectratio, width=\linewidth]{fig/ray-input.png}
    \caption{
      Rayによって3Dアプリケーションへのフォーカスを切り替えている様子.
      上段は一番左のアプリケーションに,下段は真ん中のアプリケーションにフォーカスしており,
      アプリケーションがハイライトされている様子がわかる.
    }
    \label{fig:ray-input}
  \end{minipage}
\end{figure}

\begin{figure}[htbp]
  \begin{minipage}[t]{0.50\linewidth}
    \centering
    \includegraphics[keepaspectratio, width=\linewidth]{fig/2d-apps.png}
    \caption{
      ブラウザなどの既存の2Dアプリケーションが修正なしでそのまま動作する.
    }
    \label{fig:2d-apps}
  \end{minipage}
  \begin{minipage}[t]{0.50\linewidth}
    \centering
    \includegraphics[keepaspectratio, width=\linewidth]{fig/zmonitors.png}
    \caption{
      既存の2Dアプリケーションを3Dアプリケーションに変換するzmonitors.
    }
    \label{fig:zmonitors}
  \end{minipage}
\end{figure}

\begin{figure}[htbp]
  \begin{minipage}[t]{0.50\linewidth}
    \centering
    \includegraphics[keepaspectratio, width=\linewidth]{fig/dnd.png}
    \caption{
      Google Chromeから天体を表示・編集する3Dアプリケーションに,天体のテクスチャを
      ドラッグ \& ドロップすることで,天体を地球から太陽に変更する様子.
    }
    \label{fig:dnd}
  \end{minipage}
\end{figure}
