\subsection{目標(開発面)}

\ref*{section:objective}での3つの目的に対して,本プロジェクトの未踏アドバンストにおける
開発面での目標を述べる.具体的な実装方針などについては\ref{section:dev-plan-detail}
で述べる.

\subsubsection*{実用レベルのXRデスクトップ環境を提供する.}

\begin{itemize}
  \item X11の2Dアプリケーションに対応する.\\
        現状のzmonitors(\ref{section:current-status}の3. を参照)では % TODO: 3. に変わりないかCheck
        Waylandプロトコルにのみ対応している.つまり現状ZIGENではWayland対応の
        2Dアプリケーションしか動作しない.WaylandはUbuntu 21.04からはX11に変わって
        標準のWindowing Systemとなるなど,X11より新しいWindowing Systemであると
        % textlint-disable
        言えるが,現状多くのLinuxのGUIアプリケーションはX11でのみ動作し,それらの
        % textlint-enable
        アプリケーションが使えなければ実生活での作業空間としてZIGENを利用することは厳しい.
        % TODO: Check 単にXWaylandに対応するだけの話であることを言えてるか

  \item XR デスクトップ環境としての機能の充実する.\\
        実用的な2D デスクトップ環境ではWIMPのパラダイムのもとアイコンのクリックによって
        アプリケーションを実行するといった機能が存在する.またショートカットキーによる
        アプリケーション一覧の表示や,フォーカスの切り替えなどは
        正確にはWindowing Systemとしては必須の機能ではないが,実用的なデスクトップ環境には
        求められる機能である.

  \item 2D Windowing Systemとの親和性を高める.\\
        XR空間を作業環境として使ってもらう場合でも,全てがHMDを着用したXR空間内で
        完結することには,すぐにはならないだろう.例えば,HMDが何かしらの理由で使えない場合は
        通常のディスプレイで作業することになるであろうし,作業環境を既存の
        2D デスクトップ環境に戻して,HMDをゲーム用途に使いたくなる可能性もある.
        そのため実用的なXRでは 今までと同様の2D デスクトップ環境で作業しているところから,
        起動中の2DアプリケーションなどはそのままにXR デスクトップ環境にスムーズ移行し,
        またXRデスクトップ環境から通常の2Dデスクトップ環境に戻ってこれるようにするべきである.
\end{itemize}

\subsubsection*{ユーザと開発者からなるエコシステムを形成する.}

\begin{itemize}
  \item ゲームエンジンでZIGENの3Dアプリケーションを作成できるようにする.\\
        VRやARに興味があり,開発をしているひとの多くはUnityやUnrealといったゲームエンジンで
        開発をおこなっている.現状ではZIGEN上のアプリケーションを作成するには
        OpenGLに似たAPIを叩いて描画する必要があり,多くの開発者を巻き込める状況にない.

  \item ZIGENのアプリケーションのストアを展開する.\\
        開発者が自分が作ったアプリケーションを出品し,ユーザが自分の欲しいアプリケーションを
        検索してインストールできるストアを運営することにより,ユーザと開発者とを繋ぐハブとして
        エコシステムの活性化を図る.
        しかし,これはかなり応用的な話で,未踏アドバンストにおいてできるかはわからないが,
        継続的な開発のための収益化にもつながりうるため,視野にいれている.
\end{itemize}

\subsubsection*{多くのユーザに使ってもらう}

\begin{itemize}
  \item より多くのHMDへの対応\\
        より多くのユーザに使ってもらうために,サポートするデバイスを増やす.
        特にOculus Quest2の販売台数は他の全VRヘッドセットの販売台数を上回るとも
        言われており,対応することでより多くのユーザに使ってもらう機会を増やすことができると
        考えている.

  \item 既存のXRアプリケーションと一緒に使えるようにオーバーレイとして実装する.
        現在でもVRChatなど特定のXRアプリケーションを常時使い続けている人は多い.
        このような人たちをユーザとして取り込むために,他のXRアプリケーションに対して
        オーバーレイとしてZIGENの環境を使えるようにすることで,VRChatでコミュニケーションを
        とりながら,よりよい個人の作業空間を展開するなどといったことができるようになる.
\end{itemize}

\subsection{目標(社会へのサービス提供の面)}

社会へのサービス提供の観点では特に\textgt{"ユーザと開発者からなるエコシステムを形成する"}
と\textgt{"多くのユーザに使ってもらう"}の目的に対して取り組む.
% TODO: またサービスの継続的な開発のための事業化(収益化)の観点を入れる.

\subsubsection{ユーザと開発者からなるエコシステムを形成する.}

\begin{itemize}
  \item 開発者ドキュメントを整備する.
  \item Discord などによるコミュニティーを広げる. % アプリケーションの数 / DiscordでのInquiry数
\end{itemize}

\subsubsection{多くのユーザに使ってもらう}

\begin{itemize}
  \item イベントなどの登壇を通してプレセンスを高める.% discord や Twitter のFollower
\end{itemize}

% TODO: 収益化に関するetc
