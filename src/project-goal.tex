\subsection{目標(開発面)}

\ref*{section:objective}での3つの目的に対して,本プロジェクトの未踏アドバンストにおける
開発面での目標を述べる.

\subsubsection*{実用レベルのXRデスクトップ環境を提供する.}

\begin{itemize}
  \item 2DアプリケーションのX11対応.\\
        現状のzmonitors(\ref{section:current-status}の3. を参照)では % TODO: 3. に変わりないかCheck
        Waylandプロトコルにのみ対応している.つまり現状ZIGENではWayland対応の
        2Dアプリケーションしか動作しない.WaylandはUbuntu 21.04からはX11に変わって
        標準のWindowing Systemとなるなど,X11より新しいWindowing Systemであると
        % textlint-disable
        言えるが,現状多くのLinuxのGUIアプリケーションはX11でのみ動作し,それらの
        % textlint-enable
        アプリケーションが使えなければ実生活での作業空間としてZIGENを利用することは厳しい.
        % TODO: Check 単にXWaylandに対応するだけの話であることを言えてるか

  \item 3D デスクトップ環境の充実.\\
        実用的な2D デスクトップ環境ではWIMPのパラダイムのもとアイコンのクリックによって
        アプリケーションを実行するといった機能が存在する.またショートカットキーによる
        アプリケーション一覧の表示や,フォーカスの切り替えなどは
        正確にはWindowing Systemとしては必須の機能ではないが,実用的なデスクトップ環境には
        求められる機能である.
  \item 2D Windowing Systemとの親和性を高める.
\end{itemize}

\subsubsection*{ユーザと開発者からなるエコシステムを形成する.}

\begin{itemize}
  \item ゲームエンジンによって3Dアプリケーションを作成できるようにする.
  \item ストアへ出品,またストアからインストールして使えるようにする.(want)
\end{itemize}

\subsubsection*{多くのユーザに使ってもらう}

\begin{itemize}
  \item より広いHMDへの対応
  \item 既存のXRアプリケーションと一緒に使えるようにオーバーレイとして実装する.
\end{itemize}

\subsection{目標(社会へのサービス提供の面)}

社会へのサービス提供の観点では特に\textgt{"ユーザと開発者からなるエコシステムを形成する"}
と\textgt{"多くのユーザに使ってもらう"}の目的に対して取り組む.
% TODO: またサービスの継続的な開発のための事業化(収益化)の観点を入れる.

\subsubsection{ユーザと開発者からなるエコシステムを形成する.}

\begin{itemize}
  \item 開発者ドキュメントを整備する.
  \item Discord などによるコミュニティーを広げる. % アプリケーションの数 / DiscordでのInquiry数
\end{itemize}

\subsubsection{多くのユーザに使ってもらう}

\begin{itemize}
  \item イベントなどの登壇を通してプレセンスを高める.% discord や Twitter のFollower
\end{itemize}

% TODO: 収益化に関するetc
