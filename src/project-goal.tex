\subsection{目標(開発面)}
\label{section:dev-goal}

\ref*{section:objective}の目的のうち開発に関連するものに対して,
本プロジェクトの未踏アドバンストにおける開発面での目標を述べる.
具体的な実装方針などについては\ref{section:dev-plan-detail}で述べる.

\subsubsection*{実用レベルのXRデスクトップ環境を提供する.}

\begin{itemize}
  \item \textbf{X11の2Dアプリケーションに対応する.}\\
        現状のzmonitors(\ref{section:current-status}の3. を参照)では % TODO: 3. に変わりないかCheck
        Waylandプロトコルにのみ対応している.つまり現状ZIGENではWayland対応の
        2Dアプリケーションしか動作しない.WaylandはUbuntu 21.04からはX11に変わって
        標準のWindowing Systemとなるなど,X11より新しいWindowing Systemであると
        % textlint-disable
        言えるが,現状多くのLinuxのGUIアプリケーションはX11でのみ動作し,それらの
        % textlint-enable
        アプリケーションが使えなければ実生活での作業空間としてZIGENを利用することは厳しい.
        % TODO: Check 単にXWaylandに対応するだけの話であることを言えてるか
        さらにX11の2Dアプリケーションへの対応に加えて,WineHQ\footnote{WineHQ - Run Windows applications on Linux, BSD, Solaris and macOS:\url{https://www.winehq.org/}}などを用いることで既存のWindows OS上で動作する2DアプリケーションもZIGEN上で使える可能性がある.
  \item \textbf{XR デスクトップ環境としての機能の充実する.}\\
        実用的な2D デスクトップ環境ではWIMPのパラダイムのもとアイコンのクリックによって
        アプリケーションを実行するといった機能が存在する.またショートカットキーによる
        アプリケーション一覧の表示や,フォーカスの切り替えなどは
        正確にはWindowing Systemとしては必須の機能ではないが,実用的なデスクトップ環境には
        求められる機能である.

  \item \textbf{2D Windowing Systemとの親和性を高める.}\\
        XR空間を作業環境として使ってもらう場合でも,全てがHMDを着用したXR空間内で
        完結することには,すぐにはならないと予想される.例えば,HMDが何かしらの理由で使えない場合は
        通常のディスプレイで作業することになるであろうし,作業環境を既存の
        2D デスクトップ環境に戻して,HMDをゲーム用途に使いたくなる可能性もある.
        ZIGENを通じて2D Windowing Systemを起動できるようにすれば,モニターで
        起動中の2D デスクトップ環境からXR デスクトップに移行する際に, 2D デスクトップのアプリケーションを
        そのままXR デスクトップに移すことができる.逆もまた然りでXR デスクトップから通常の2Dのデスクトップ環境に戻ることができる.
\end{itemize}

\subsubsection*{ユーザと3Dアプリケーション開発者からなるエコシステムを形成する.}

\begin{itemize}
  \item \textbf{ゲームエンジンでZIGENの3Dアプリケーションを作成できるようにする.}\\
        VRやARに興味があり,開発をしているひとの多くはUnityやUnrealといったゲームエンジンで
        開発をおこなっている.現状ではZIGEN上のアプリケーションを作成するには
        OpenGLに似たAPIを叩いて描画する必要があり,多くの開発者を巻き込める状況にない.

  \item \textbf{ZIGENのアプリケーションのストアを展開する.}\\
        開発者が自分が作ったアプリケーションを出品し,ユーザが自分の欲しいアプリケーションを
        検索してインストールできるストアを運営することにより,ユーザと開発者とを繋ぐハブとして
        エコシステムの活性化を図る.
        しかし,これはかなり応用的な話で,未踏アドバンストにおいてできるかはわからないが,
        継続的な開発のための収益化にもつながりうるため,視野にいれている.
\end{itemize}

\subsubsection*{多くのユーザに使ってもらう.}

\begin{itemize}
  \item \textbf{より多くのHMDへの対応}\\
        より多くのユーザに使ってもらうために,サポートするデバイスを増やす.
        特にOculus Quest2の販売台数は他の全VRヘッドセットの販売台数を上回るとも
        言われており,対応することでより多くのユーザに使ってもらう機会を増やすことができると
        考えている.

  \item \textbf{既存のXRアプリケーションと一緒に使えるようにオーバーレイとして実装する.}\\
        現在でもVRChatなど特定のXRアプリケーションを常時使い続けている人は多い.
        このような人たちをユーザとして取り込むために,他のXRアプリケーションに対して
        オーバーレイとしてZIGENの環境を使えるようにすることで,VRChatでコミュニケーションを
        とりながら,よりよい個人の作業空間を展開するなどといったことができるようになる.
\end{itemize}

\subsection{目標(社会実装面)}

\ref{section:objective}の目的のうち社会実装に関連するものに対して,
本プロジェクトの未踏アドバンストにおける社会実装面での目標を述べる.
% 具体的な施策に関しては\ref{section:biz-plan-detail}で述べる.

ここで事業性の面で本プロジェクトが主に目指すところは,収益化ではなく,
現状のXRにおける課題を解決し,作業空間としてのXRのあるべき世界を指し示し,
社会にインストールすることである.ただし,プロジェクトの継続的な発展や開発母体の安定性のために,
収益化の方法を模索することは必要であると思っており,未踏アドバンスト事業期間中にそのための
検証などを行なっていくつもりである.

\subsubsection*{ユーザと3Dアプリケーション開発者からなるエコシステムを形成する.}

% \item 開発者ドキュメントを整備する.
まずは開発や利用に関して気軽に質問などができるコミュニティを目指し,
Discordなどのコミュニティツールの整備や,その広報をおこなっていく.
ターゲットユーザの近い類似プロジェクトの現状を鑑みて以下の目標数値を定める.

\begin{itemize}
  \item Discordユーザ:300人 (現状36人)
  \item 外部からのIssue/PullRequest:5つ
\end{itemize}

また,外部の開発者に実際にアプリケーションを作ってもらい試してもらうためのハッカソンを
開催する.

\subsubsection*{多くのユーザに使ってもらう.}

幅広い知識レベルのユーザが利用できるようにインストラクションドキュメントを提供すると共に,
カンファレンスやイベント,メディアへの露出や,ブログによる情報発信を通して認知度の向上を図る.

\subsubsection*{XR世界を作り上げているコミュニティでのプレゼンスを高める.}

OSSとしてのコミュニティの完成度を高めるため.
GitHubが提供しているOSSプロジェクトのガイドライン
\footnote{オープンソースガイドライン:https://opensource.guide/ja/}
を実践する.
また,先に述べたカンファレンスやイベントなどの登壇もプレゼンスの向上に寄与すると考える.

\subsubsection*{事業の継続のための収益化について.}

開発継続のための金銭を受け取る方法はいくつか模索している.

\begin{itemize}
  \item \textbf{助成金やクラウドファウンディングなどで金銭を得る\\}
        OSS向けのFoundationは多く存在するが,その中でも特に
        XR Foundation\footnote{XR Foundation:https://www.xrfoundation.io/}や,
        Open 3D Foundation\footnote{Open 3D Foundation:https://o3d.foundation/}
        などのXR向けのFoundationに絞って申請を検討する.また,Webpack\footnote{Webpack:https://webpack.js.org/}などはOpenCollective
        \footnote{webpack - OpenCollective https://opencollective.com/webpack}
        を通じて企業や個人から資金を得ていたり,Kickstarterのキャンペーンを通じてDjangoの
        スキーママイグレーションの活動資金を得ているプロジェクト
        \footnote{https://www.kickstarter.com/projects/andrewgodwin/schema-migrations-for-django}
        なども存在する.
  \item \textbf{ビジネスとして展開する.\\}
        ZIGENにおいてビジネス展開を考える場合はHMDやLinux PCを持っている人の絶対数が
        多くはないためそこが難しい.そのため,ZIGENをセットアップした状態のハードウェア
        含めたXR作業環境一式,またはその一部(HMDとWindowing Systemのみなど)
        を法人向けに提供することを考えている.昨今はクラウドサービスが増え,
        Google SheetのようにOSに依存せず,ブラウザさえあれば仕事が完結する場合も
        多くなってきており,Linuxのデスクトップ環境でも十分に仕事が可能になってきている.
        とくにエンジニア向けにエンジニアを抱える法人に対してアプローチしていきたい.
\end{itemize}

未踏アドバンスト事業期間での目標は,助成金またはクラウドファンディングなどにはいずれか1つ以上
申請し,ビジネス展開については普段から仕事でLinuxを使っている開発者,そうでない開発者,
非開発者などの属性それぞれに対して,ZIGENでの作業環境に移行する可能性があるか
% textlint-disable
ユーザインタビューを行い検証していきたい.
% textlint-enable

% TODO: 収益化に関するetc
