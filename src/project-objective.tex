\subsection{目的}
\label{section:objective}

本プロジェクトの目的はXR空間にWindowing Systemを導入することで
XR空間にマルチアプリケーション・マルチタスクの環境を提供することである.
\ref{section:current-status}で述べたように,本プロジェクトは未踏IT人材発掘・育成事業
% textlint-disable
において,メインのコンセプトについて概念実証を行ってきが,これをさらに社会へ実装するため,
% textlint-enable
未踏アドバンストでは以下のより詳細な目的を設定する.

\begin{enumerate}
  \item 実用レベルのXRデスクトップ環境を提供する.\\
        ユーザが日常で使うデスクトップ環境を
        提供するためには,単純にディスプレイサーバを実装するだけでは足りず,
        デスクトップ環境などと呼ばれる部分の実装などが必要になる.
  \item ユーザと3Dアプリケーション開発者からなるエコシステムを形成する.\\ %TODO: ここは伝わる言い方になってない。改善する。
        本プロジェクトが目指す世界の特徴として,XR空間全体ではなく,各機能がそれぞれ
        アプリケーションとなることで,機能ごとに市場競争が生じ,
        また開発者も特定の機能だけをうまく実装すればよいことになる.
        これによって現在より多様な機能や多様な開発者が生まれ,作業空間としてのXRは
        より洗練された空間になる.この世界の実現のため,まずは個人の開発者が空間の改善を行え,
        それをユーザに使ってもらえる仕組みを提供していく.
  \item 多くのユーザに実際に試してもらう.\\
        未踏IT人材発掘・育成事業では実際にユーザに使ってもらい,生のフィードバックを
        多くもらうまでは至らなかった.これを達成するためにはイベントの参加・開催,
        対応ハードウェア,認知度向上などに関してするべきことがある.
  \item XR世界を作り上げているコミュニティでのプレゼンスを高める.\\
        XRの環境は今もまだまだ開発段階であり,多くの開発者がXR空間の改善のために活動している.
        その中でプロジェクトの価値を認められることは重要である.なぜなら,XRの環境は多くの
        技術を組み合わせた世界であり,本プロジェクトが長期的なスパンで成功するためには,
        他のOSSプロジェクトなどの協力が不可欠であるからである.(例えばよりパフォーマンス向上
        のためにはOpenGLにおいてExtensionの開発などが求められる.)
        また,本プロジェクトは技術上の制約でLinux上でしか動作しないが,他のOSに対しても
        影響を与え,XR Windowing Systemのコンセプトを広く普及させることにも繋がる.
\end{enumerate}
