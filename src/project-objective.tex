\subsection{目的}
\label{section:objective}

本プロジェクトの目的はXR空間にWindowing Systemを導入することで
XR空間にマルチアプリケーション・マルチタスクの環境を提供することである.
\ref{section:current-status}で述べたように,本プロジェクトは未踏IT人材発掘・育成事業
% textlint-disable
において,メインのコンセプトについて概念実証を行ってきが,これをさらに社会へ実装するため,
% textlint-enable
未踏アドバンストでは以下のより詳細な目的を設定する.

\begin{enumerate}
  \item 実用レベルのXRデスクトップ環境を提供する.\\
        ユーザが日常で使うデスクトップ環境を
        提供するためには,単純にディスプレイサーバを実装するだけでは足りず,
        アプリケーションラウンチャーなどのグラフィカルシェルと呼ばれる部分の実装などが必要になる.
  \item ユーザと3Dアプリケーション開発者からなるエコシステムを形成する.\\ %TODO: ここは伝わる言い方になってない。改善する。
        本プロジェクトが目指す世界の特徴として,XR空間全体ではなく各機能がそれぞれ
        アプリケーションとなることで,機能ごとに市場競争が生じ,
        また開発者も特定の機能だけをうまく実装すればよいことになる.
        これによって現在より多様な機能や多様な開発者が生まれ,作業空間としてのXRは
        より洗練された空間になる.この世界の実現のため,まずは個人の開発者が空間の改善を行え,
        それをユーザに使ってもらえる仕組みを提供していく.
  \item 多くのユーザに実際に試してもらう.\\
        未踏IT人材発掘・育成事業では実際にユーザに使ってもらい,生のフィードバックを
        多くもらうまでは至らなかった.これを達成するためにはイベントの参加・開催,
        多様なハードウェアへの対応,認知度向上などに関してするべきことがある.
  \item XR世界を作り上げているコミュニティでのプレゼンスを高める.\\
        XRの環境は今もまだまだ開発段階であり,多くの開発者がXR空間の改善のために活動している.
        その中でプロジェクトの価値を認められることはOSSとしての継続的な開発と技術の浸透の
        ために重要である.また,XRの環境は多くの技術を組み合わせた世界であり,本プロジェクトが
        長期的なスパンで成功するためには,他のOSSプロジェクトなどの協力が不可欠であり,
        その点でも重要となってくる.(例えばよりパフォーマンス向上のためにはOpenGLにおいて
        Extensionの開発などが求められる.)また,本プロジェクトは技術上の制約で
        Linux上でしか動作しないが,他のOSに対しても影響を与え,XR Windowing Systemの
        コンセプトを広く普及させることにも繋がる.
\end{enumerate}
